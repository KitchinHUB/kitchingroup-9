%% ****** Start of file template.aps ****** %
%%
%%
%%   This file is part of the APS files in the REVTeX 4 distribution.
%%   Version 4.0 of REVTeX, August 2001
%%
%%
%%   Copyright (c) 2001 The American Physical Society.
%%
%%   See the REVTeX 4 README file for restrictions and more information.
%
% Group addresses by affiliation; use superscriptaddress for long
% author lists, or if there are many overlapping affiliations.
% For Phys. Rev. appearance, change preprint to twocolumn.
% Choose pra, prb, prc, prd, pre, prl, prstab, or rmp for journal
%  Add 'draft' option to mark overfull boxes with black boxes
%  Add 'showpacs' option to make PACS codes appear
%  Add 'showkeys' option to make keywords appear
%\documentclass[aps,prl,preprint,superscriptaddress,endfloats*,showkeys]{revtex4}
\documentclass[aps,prl,twocolumn,superscriptaddress,showkeys]{revtex4}
\usepackage{graphicx}

\begin{document}

% Use the \preprint command to place your local institutional report
% number in the upper righthand corner of the title page in preprint mode.
% Multiple \preprint commands are allowed.
% Use the 'preprintnumbers' class option to override journal defaults
% to display numbers if necessary
%\preprint{}

%Title of paper
\title{The role of strain and ligand effects in the modification of
  the electronic and chemical properties of bimetallic surfaces}

% repeat the \author .. \affiliation  etc. as needed
% \email, \thanks, \homepage, \altaffiliation all apply to the current
% author. Explanatory text should go in the []'s, actual e-mail
% address or url should go in the {}'s for \email and \homepage.
% Please use the appropriate macro foreach each type of information

% \affiliation command applies to all authors since the last
% \affiliation command. The \affiliation command should follow the
% other information
% \affiliation can be followed by \email, \homepage, \thanks as well.
\author{J. R. Kitchin}
\affiliation{Center for Catalytic Science and Technology, Department of Chemical Engineering, University of Delaware,
  Newark, DE 19716}

\author{J. K. N{\o}rskov}
\affiliation{Center for Atomic-scale Materials Physics and Department
  of Physics, Technical University of Denmark, DK-2800 Lyngby,
  Denmark}

\author{M. A. Barteau}
\author{J. G. Chen}
\affiliation{Center for Catalytic Science and Technology, Department of Chemical Engineering, University of Delaware, Newark, DE 19716}

\date{\today}

\begin{abstract}
Periodic density functional calculations are used to illustrate how
the combination of strain and ligand effects modify the electronic
and surface chemical properties of Ni, Pd, and Pt monolayers
supported on other transition metals. Strain and the ligand effects
are shown to change the width of the surface $d$-band, which
subsequently moves up or down in energy to maintain a constant band
filling. Chemical properties such as the dissociative adsorption
energy of hydrogen are controlled by changes induced in the
average energy of the $d$-band by modification of the $d$-band
width.
\end{abstract}

% insert suggested PACS numbers in braces on next line
\pacs{}
% insert suggested keywords - APS authors don't need to do this
\keywords{Bimetallic surfaces, strain, ligand effect, DFT, matrix element,
  hydrogen, adsorption, $d$-band}

%\maketitle must follow title, authors, abstract, \pacs, and \keywords
\maketitle

%The chemical properties of metals can be modified by alloying them
%with another metal \cite{sinfelt1983}. 
Numerous surface science studies have shown that the chemisorption
properties of supported monolayers of one metal on another metal can
be dramatically different than those of the parent metals
\cite{goodman1990:_chemis,campbell1992:_chemic,rodriguez1992}. There
are two primary mechanisms for modification of the chemical properties
of these surfaces. First, the average bond lengths between the metal
atoms in the supported monolayer surface are typically different than
those in the parent metals, resulting in changes due to strain.
Second, heterometallic bonding interactions, termed the ``ligand
effect'', between the surface atoms and the substrate can result in
modification of the surface electronic structure, thereby changing the
surface chemical properties.  However, it has been difficult to
separate the strain and ligand effects, even in surface science
studies, because they usually occur together. In other words, one
typically observes the cumulative effect of the two phenomena.  One
notable recent success was the deconvolution of these two effects for
heteroepitaxial layers of Pt on Ru(0001) using a combination of
experiments and theory
\cite{schlapka2003:_surfac_strain_subst_inter_heter_metal_layer}, but
this work succeeded only in explaining the behavior of a single
system.


Mavrikakis \emph{et al.} used density functional (DFT) calculations to show
that strain modifies the chemical properties of surfaces by changing
the average energy of the $d$-band
\cite{mavrikakis1998:_effec_strain_metal}. DFT was also recently used
to illustrate the ligand effect, by showing that the presence of
subsurface 3d metals in Pt(111) surfaces broadens the Pt surface
$d$-band.  The width of the surface $d$-band was found to be proportional
to the interatomic matrix element that describes bonding interactions
between the $d$-orbitals of the Pt surface atoms and the $d$-orbitals of
the subsurface 3d metal atoms
\cite{kitchin2004:_modif_surfac_elect_chemic_proper}. As a consequence
of the band broadening, the average energy of the $d$-band moved down in
energy in order to conserve the $d$-band filling, resulting in
modifications of the surface chemical properties.

In the current Letter, we present a simple explanation for the
electronic modifications induced by strain and ligand effects in
bimetallic surfaces, as well as the effects of these modifications on
the chemical properties of these surfaces. We will show that the width
of the $d$-band is modified by both strain and the ligand effect, and
that the effects of these two mechanisms are cumulative.  The $d$-band
width is found to be proportional to the interatomic matrix element
describing bonding between the $d$-orbitals of an atom and the
$d$-orbitals of its nearest neighbors. The average energy of the $d$-band is
increased or decreased depending on whether the band becomes narrower
or wider due to the combination of strain and ligand effects to
maintain a constant $d$-band filling. Finally,  changes in chemical
properties of these bimetallic surfaces due to the combined strain and
ligand effects can be traced to changes in the average energy of the
surface $d$-band associated with the changes in $d$-band width.

The supported monolayer calculations were performed using DACAPO on
three layer slabs of the closest-packed surface for the substrate crystal
structure (fcc(111), bcc(110), or hcp(0001).) The surface
layer of the slab was replaced with the monolayer metal in the bulk
truncated position.  In our notation, PtW designates a pseudomorphic
monolayer of Pt on a W(110) surface, NiRu designates a pseudomorphic
Ni monolayer on Ru(0001), and NiPt designates a pseudomorphic Ni
monolayer on Pt(111).  Relaxation of the slab was allowed in some
cases, and the effects of this are discussed later.  The slabs were
periodically repeated in $2\times2$ super cells, and the ionic cores
were represented by Vanderbilt ultrasoft pseudopotentials.  The
Kohn-Sham one-electron orbitals were expanded using planewaves up to a
350 eV cutoff energy.  The Brillouin zone was sampled either by a
Chadi-Cohen grid of 18 k-points or a $4\times4\times1$ Monkhorst-Pack
grid. The trends and conclusions in this work were independent of this
choice.  The PW91 functional was used to describe exchange and
correlation. Properties of the $d$-band were calculated by projecting
the Kohn-Sham orbitals onto spherical harmonics centered on the atomic
sites.

We first examine whether the formation of the bimetallic surfaces
alters the filling of the surface $d$-band. Changes in the $d$-band
filling would indicate charge transfer between the two metals, and
could be one of the mechanisms for chemical modification.  In all
cases the change in $d$-band filling was typically less than 2\% in
magnitude on formation of the bimetallic surface. We interpreted these
as negligible changes.  Furthermore, the magnitude and sign of the
change is basis set dependent; the magnitudes decrease and the sign
can change if a cutoff radius is used to limit overcounting from
neighboring sites. Thus, these changes may not be meaningful, aside
from the observation that they are small. We also found that the
$d$-band filling does not change when strain is artificially introduced
 despite strain-induced changes in the $d$-band width.

As shown for a rectangular band model, if the $d$-band filling remains
constant but the $d$-band width changes, then the average energy of the
$d$-band (the $d$-band center) must change to conserve both the $d$-band
filling and the total number of $d$-states
\cite{kitchin2004:_modif_surfac_elect_chemic_proper}. Figure
\ref{fig:momentcorrelations} shows the correlation between the root
mean squared (rms) $d$-band width and the $d$-band center for both
unrelaxed and relaxed slabs. The slope of the correlation is
nearly -1, as predicted by the rectangular band
model. Thus, as the $d$-band becomes wider it moves down in energy,
and as it narrows it moves up in energy. The non-zero intercept is
likely an artifact of the gaussian broadening used in the projection
of the density of states onto atomic orbitals to account for the
finite number of k-points used in the calculations. This prevents the
$d$-band width from going to zero in the atomic limit as it should.

\begin{figure}
\includegraphics{Fig1}
\caption{Correlation between the root mean squared (rms) $d$-band width
  and $d$-band center for bimetallic surfaces.\label{fig:momentcorrelations}}
\end{figure}

The width of the $d$-band can be related to the interatomic matrix
element that describes bonding between an atom and its environment
\cite{harrison1989}.  The matrix element describing the bonding
interactions between the $d$-states on two atoms
of type 1 and type 2 is given by
\cite{harrison1989}
%
\begin{equation}\label{eq:v}
V^{(1,2)}_{ddm}=\eta_{ddm}\frac{\hbar^2 (r_d^{(1)}r_d^{(2)})^{3/2}}{md_{1,2}^5},
\end{equation}
%
where $m$ is the mass of an electron, and $r_d^{(i)}$ is a
characteristic length for metal $i$ that is related to the spatial
extent of the $d$-orbitals of that metal, which was obtained from the
Solid State Table in Reference \citealp{harrison1989}.  The general
trends in this Table show that $r_d^{(i)}$ increases from right to
left and from top to bottom among the transition metals in the
Periodic Table.  Alternatively, $r_d^{(i)}$ can be shown to be
proportional to the linear muffin tin orbital (LMTO) potential
parameters by $r_d^{(i)} \propto M_{id}^{2/3} \propto (s_i^5
\Delta_d^{Andersen})^{1/3}$ \cite{jacobsen1987:_inter,
  ruban1997:_surfac}. Here $M_{id}$ is an intra-atomic matrix element,
$s_i$ is the experimental Wigner-Seitz radius and
$\Delta_d^{Anderson}$ is a potential parameter, both of which can be
found in Ref.  \cite{andersen1985:_canon_descr_band_struc_metal}.
Ruban \emph{et al.}  reported coupling matrix elements $V_{ad}^2$, which are
proportional to $M_{id}^2$, between an adsorbate state and a metal
$d$-state for all of the transition metals \cite{ruban1997:_surfac}.
Thus, Eq.  (\ref{eq:v}) is seen to be equivalent to Eq. 6 in Ref.
\citealp{ruban1997:_surfac}, but applied specifically to the coupling
between two $d$-states.  $d_{1,2}$ is the internuclear distance between
atoms $1$ and $2$.  $\eta_{ddm}$ is a dimensionless proportionality
constant that is formally related to the type of $dd$ orbital bonding
interaction, such as $V_{dd\sigma}$, $V_{dd\pi}$, or $V_{dd\delta}$.
We are primarily interested in the proportionality between the
interatomic matrix element and the $d$-band width, so we take
$\eta_{ddm}$ to be unity for convenience, and thus treat all of the
possible interactions in a lumped manner.

Eq.  (\ref{eq:v}) was previously used to describe the ligand effect in
Pt-3d sandwich structures
\cite{kitchin2004:_modif_surfac_elect_chemic_proper}. In that work,
$d_{12}$ represented the interlayer spacing between the surface and
subsurface layers. In this work, we expand upon this idea, recognizing
that the ligand effect is incorporated into the numerator of Eq.
(\ref{eq:v}) through the $r_d^{(i)}$ of each metal, and that strain is
incorporated into the denominator through the distance between the
atoms.  We further develop the model by treating the interactions of a
surface atom with all of the nearest-neighbors individually, including
the subsurface nearest neighbors, so that the matrix element
associated with a given surface atom is given by
%
\begin{equation}\label{eq:vij}
V_{i}=7.62\sum_{j=1}^{NN} \frac{(r_d^{(i)}r_d^{(j)})^{3/2}}{d_{ij}^5},
\end{equation}
%
where $\hbar^2/m=7.62$ eV\AA$^2$.

According to tight binding theory, the $d$-band width should be
proportional to the matrix element given by Eq. (\ref{eq:vij}).  This
is illustrated in Figure \ref{fig:vw}, where the rms $d$-band width of a
large number of geometries of both pure and bimetallic Ni, Pd, and
Pt-containing structures is plotted against their respective
interatomic matrix elements. These structures include a variety of
forms of strain, including isotropic and anisotropic strain,
pseudomorphic monolayers of Ni, Pd, and Pt on Nb, Ta, Mo, W, Re, Ru,
Ir, Os, Ni and Pt, surface and subsurface alloys (from Ref.
\cite{kitchin2003:_elucid_ni_pt}), bimetallic sandwich structures
(from Ref. \cite{kitchin2004:_modif_surfac_elect_chemic_proper}), and
a variety of coordination environments (sheets, slabs, and solids).
Clearly, the $d$-band width is closely related to the interatomic matrix
element that describes the bonding interactions between the $d$-orbitals
of the atom and its nearest neighbors. We can now provide a
quantititative understanding of the role of strain. In tensile strain,
$d_{ij}$ is larger than in the parent metal, resulting in a smaller
interatomic matrix element, and a narrower $d$-band that is higher in
energy than the parent metal. In compressive strain, $d_{ij}$ is
smaller than that of the parent metal, resulting in a larger
interatomic matrix element, and a wider $d$-band that is lower in energy
than that of the parent metal.
 
\begin{figure}
\includegraphics{Fig2}
\caption{Correlation between the rms $d$-band width and the matrix
  element. Black symbols indicate a Ni atom, grey symbols indicate Pd,
  and white indicates Pt. Data points labeled NiPt2.xx refer to Ni/Pt
  bimetallic surfaces with nearest neighbor distances of 2.77 \AA{} or
  2.83 \AA \cite{kitchin2003:_elucid_ni_pt}. Details for the Pt-3d
  surfaces can be found in Ref.
  \citealp{kitchin2004:_modif_surfac_elect_chemic_proper}. The keyword
  Uniform indicates the structures were strained uniformly in the unit
  cell lattice vector directions, otherwise, the structure was
  strained only in one lattice vector direction. A sheet is a single
  atomic layer, a slab is three layers thick and separated by five
  equivalent layers of vacuum. The monolayers are described in the
  text.
  \label{fig:vw}}
\end{figure}

We conclude by showing how these electronic structure modifications
are manifested in modifying the chemical properties of bimetallic
surfaces. The adsorption energies and dissociative reaction barriers
of many small molecules such as CO have been correlated with the
average energy of the surface $d$-band of pure metal surfaces
\cite{norskov2002:_univer} and of alloy/overlayer surfaces
\cite{hammer1996:_co}. We have calculated the dissociative adsorption
energies for hydrogen on many of the bimetallic monolayer systems
previously described in this work. In all of these calculations, the
bimetallic slab was allowed to relax to the lowest energy geometry.
The positions of the slab ions were subsequently frozen in their
relaxed positions, and a H atom was placed in a three-fold site and
allowed to relax to the lowest energy geometry. The dissociative
adsorption energy was calculated as $\Delta
H_{diss,ads}=E_{slab,H}-E_{slab}-1/2E_{H_2}$.  The dissociative
adsorption energies are plotted against the $d$-band center of the
surfaces in Figure \ref{fig:HBE}, where an obvious trend between the
dissociative adsorption energy of hydrogen and the surface $d$-band
center is observed. Thus, the chemical properties of these bimetallic
surfaces, which include both strain and ligand effects, are modified
by changes in the $d$-band center of the surface. We would like to note
that a better correlation between hydrogen adsorption energies and the
$d$-band center can be obtained if a cutoff radius is introduced in the
projected density of states. However, the cutoff radius is arbitrary,
so we have not presented these results.

\begin{figure}
\includegraphics{Fig3}
\caption{Correlation between the dissociative adsorption energy of
  H$_2$ on bimetallic surfaces and the $d$-band center of those
  surfaces. \label{fig:HBE}}
\end{figure}

Figure \ref{fig:HBE} contains bimetallic surfaces in which strain and
the ligand effect work in opposite directions. For example, a Ni
monolayer on W(110) is under substantial tensile strain ($>10\%$)
compared to Ni(111), resulting in a narrowing of the $d$-band. However,
there is a substantial broadening of the Ni $d$-band due to the strong
bonding interactions with the W substrate.  The net result is a Ni
$d$-band that is broader and lower in energy than that of Ni(111) and
that the dissociative adsorption energy of H$_2$ on Ni/W(110) is lower
than that on Ni(111). For a Ni monolayer on Ru(0001), the tensile
strain is similar to that on W(110), but the ligand effect is weaker.
Consequently, the band sharpening effect due to strain is nearly
balanced by the broadening due to the ligand effect, and the
calculated electronic and chemical properties of Ni/Ru(0001) are
nearly identical to those of Ni(111). Finally, for a Ni monolayer on
Pt(111), the tensile strain is again similar to that on W(110) or
Ru(0001), but here the ligand effect is the weakest.  Consequently,
strain dominates the modification with the net effect of narrowing and
raising in energy of the Ni $d$-band, which results in a stronger
adsorption energy on this surface. The interatomic matrix elements for
each Ni surface and the corresponding $d$-band widths are tabulated in
Table \ref{tab1}. If one uses these matrix elements to predict the
band widths using the fit in Fig. \ref{fig:vw}, one finds the
bimetallic band widths are overestimated by 10-20\%. The reason for
this is clear from Fig. \ref{fig:vw}, in which nearly all of the Ni-containing
points fall below the best fit line. This suggests that
element-specific fits may give better predictive results.

\begin{table}%[H] add [H] placement to break table across pages
\caption{\label{tab1}Comparison of interatomic matrix elements and the
  DFT $d$-band width.}
\begin{ruledtabular}
\begin{tabular}{ccccc}
\hline
     &  V  &  W$_{DFT}$  \\\hline
Ni   & 0.23 & 1.89  \\\hline
NiPt & 0.17 & 1.35  \\\hline
NiRu & 0.27 & 1.76  \\\hline
NiW  & 0.28 & 2.00  \\\hline
\end{tabular}
\end{ruledtabular}
\end{table}

Finally, the trends reported in this work hold for both
unrelaxed and relaxed calculations. Allowing slabs to relax had a
dramatic effect on the electronic structure and adsorption energies in
some cases, but we found that both properties changed consistently,
effectively moving up or down the trends accordingly. These findings
are easily understood by realizing that relaxation of the slab amounts
to changing the internuclear distances of the atoms in the slab, and
that these changes are accounted for by the interatomic matrix
elements in the denominator of Eq. (\ref{eq:vij}). Thus, upon
relaxation the interatomic matrix elements change, resulting in a
change in the $d$-band width with a concomitant change in the $d$-band
center and corresponding adsorption energy. A similar analysis could be
done to estimate the effects that a surface reconstruction could have on
chemical reactivity.

In conclusion, the formation of a bimetallic surface results in
changes in the surface $d$-band width due to the combination of strain
and ligand effects.  Both of these effects are manifested in the
interatomic matrix element describing bonding interactions between an
atom and its nearest neighbors. The $d$-band width and $d$-band center are
highly correlated to each other due to the fact that the $d$-band
filling changes negligibly upon the formation of these bimetallic
surfaces.  Consequently, when the combined effects result in a broader
$d$-band, the average energy of the $d$-band decreases. In contrast, when
the combined effects result in a more narrow $d$-band, the average
energy of the $d$-band is increased. Finally, a reasonable correlation
between the dissociative adsorption energy of hydrogen and the surface
$d$-band center has been demonstrated.  The correlations described in the
current Letter could be used as a first step towards predicting the
properties of other bimetallic systems or identifying bimetallic
systems with desirable electronic or chemical properties without the
need to perform expensive first-principles DFT calculations.  This
could facilitate narrowing the focus of future studies on bimetallic
or alloy surfaces.



\begin{acknowledgments}
  This work was funded in part by Basic Energy Sciences, Department of
  Energy (Grant DE-FG02-04ER15501).  The Center for Atomic-scale
  Materials Physics is sponsored by the Danish National Research
  Foundation. The DFT calculations have been performed with support
  from the Danish Center for Scientific Computing through grant no.
  HDW-1101-05.
\end{acknowledgments}

% Create the reference section using BibTeX:
\bibliography{v_w}

\end{document}
%
% ****** End of file template.aps ******

